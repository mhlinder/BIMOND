\documentclass{article}

\usepackage{fullpage,amsmath}
\setcounter{MaxMatrixCols}{20}

\begin{document}
{
  \centering
  Derivation of Bicubic Interpolation Polynomial \\
  M. Henry Linder \\[.75em]
}

The BIMOND algorithms, given function values $f(x_i, y_j)$ for the
input values $x_i$, $i = 1, \dots, n_x$, and $y_j$,
$j = 1, \dots, n_y$, provides the values of the derivatives $f_x$,
$f_y$, and $f_{xy}$ at these input points. It is clear that we can
write \cite{wikibicubic} the interpolated surface and its derivatives
as

\begin{align*}
  f(x, y) &= \sum_{i=0}^3\sum_{j=0}^3 a_{ij}x^iy^j \\
  f_x(x,y) &= \sum_{i=1}^3 \sum_{j=0}^3 a_{ij}ix^{i-1}y^j \\
  f_y(x, y) &= \sum_{i=0}^3 \sum_{j=1}^3 a_{ij}x^ijy^{j-1} \\
  f_{xy}(x, y) &= \sum_{i=1}^3 \sum_{j=1}^3 a_{ij}ix^{i-1}jy^{y-1}
\end{align*}

Let the corners of the interpolated surface be at $f(0, 0)$,
$f(h, 0)$, $f(0, k)$, and $f(h, k)$; let the coefficients and function
values be written as

\begin{align*}
  \alpha =& (a_{00} \quad a_{10} \quad a_{20} \quad a_{30} \quad a_{01}
           \quad a_{11} \quad a_{21} \quad a_{31} \quad a_{02} \quad a_{12}
           \quad a_{22} \quad a_{32} \quad a_{03} \quad a_{13} \quad a_{23}
           \quad a_{33})^T \\
  x =& (f(0,0) \quad f(h,0) \quad f(0,k) \quad f(h,k) \quad f_x(0,0)
      \quad f_x(h,0) \quad f_x(0,k) \quad f_x(h,k) \\
  & f_y(0,0) \quad f_y(h,0) \quad f_y(0,k) \quad f_y(h,k) \quad f_{xy}(0,0) \quad f_{xy}(h,0) \quad f_{xy}(0,k) \quad f_{xy}(h,k))^T
\end{align*}

such that we can write the derivatives and function values as a system
of equations $A\alpha = x$, where

\[
A = \begin{bmatrix}
  % f
  1 & 0 & 0    & 0     & 0 & 0 & 0 & 0 & 0    & 0 & 0 & 0 & 0    & 0 & 0 & 0 \\
  1 & h & h^2 & h^3 & 0 & 0 & 0 & 0 & 0    & 0 & 0 & 0 & 0    & 0 & 0 & 0 \\
  1 & 0 & 0    & 0     & k & 0 & 0 & 0 & k^2 & 0 & 0 & 0 & k^3 & 0 & 0 & 0 \\
  1 & h & h^2 & h^3 & k & hk & h^2k & h^3k & k^2 & hk^2 & h^2k^2 & h^3k^2 & k^3 & hk^3 & h^2k^3 & h^3k^3 \\
  % f_x
  0 & 1 & 0    & 0     & 0 & 0 & 0 & 0 & 0    & 0 & 0 & 0 & 0    & 0 & 0 & 0 \\
  0 & 1 & 2h & 3h^2 & 0 & 0 & 0 & 0 & 0    & 0 & 0 & 0 & 0    & 0 & 0 & 0 \\
  0 & 1 & 0   & 0      & 0 & k & 0 & 0 & 0    & k^2 & 0 & 0 & 0 & k^3 & 0 & 0 \\
  0 & 1 & 2h & 3h^2 & 0 & k & 2hk & 3h^2k & 0 & k^2 & 2hk^2 & 3h^2k^2 & 0 & k^3 & 2hk^3 & 3h^2k^3 \\
  % f_y
  0 & 0 & 0 & 0 & 1 & 0 & 0 & 0 & 0    & 0 & 0 & 0 & 0    & 0 & 0 & 0 \\
  0 & 0 & 0 & 0 & 1 & h & h^2 & h^3 & 0 & 0 & 0 & 0 & 0 & 0 & 0 & 0 \\
  0 & 0 & 0 & 0 & 1 & 0 & 0 & 0 & 2k & 0 & 0 & 0 & 3k^2 & 0 & 0 & 0 \\
  0 & 0 & 0 & 0 & 1 & h & h^2 & h^3 & 2k & 2hk & 2h^2k & 2h^3k & 3k^2 & 3hk^2 & 3h^2k^2 & 3h^3k^3 \\
  % f_{xy}
  0 & 0 & 0 & 0 & 0 & 1 & 0 & 0 & 0    & 0 & 0 & 0 & 0    & 0 & 0 & 0 \\
  0 & 0 & 0 & 0 & 0 & 1 & 2h & 3h^2 & 0    & 0 & 0 & 0 & 0    & 0 & 0 & 0 \\
  0 & 0 & 0 & 0 & 0 & 1 & 0 & 0 & 0 & 2k & 0 & 0 & 0 & 3k^2 & 0 & 0 \\
  0 & 0 & 0 & 0 & 0 & 1 & 2h & 3h^2 & 0 & 2k & 4hk & 6h^2k & 0 & 3k^2 & 6hk^2 & 9h^2k^2 \\
  \end{bmatrix}
\]

\begin{thebibliography}{9}
  \bibitem{wikibicubic} \emph{Bicubic interpolation}. (2015, August
    4). Retrieved from \texttt{https://en.wikipedia.org/wiki/Bicubic\_interpolation}.
\end{thebibliography}

\end{document}